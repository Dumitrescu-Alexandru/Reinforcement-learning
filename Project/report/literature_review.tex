Although the game of Pong is much simpler than traditional board games such as chess, the number of states in a game are still exponentially large. This is due to the position of both the players' paddle and the location of the ball. Hence we can conclude that the table based methods for Q-learning or SARSA. 

\medskip
\noindent
We then turn our attention to approximation representation methods, which seek to use neural networks to approximate the Q-value table in Deep Q-networks or the policy and critic values in the actor critic method. This also increases the sample frequency of the methods compared to tabular methods and allow us to represent large state spaces.

\medskip
\noindent
In this project we choose to implement DQNs that allow us to use neural networks outputs to control the choices our agent makes hence allowing us to have a single forward pass to compute the effective Q-value for all the actions in the action space. This concept was introduced in \cite{Atari_Breakout} along with a few improvements such as experience replay memory which allows minibatch updates of the Q-learning algorithm. The advantage of such approaches is that it is model free and and agent should have high generalizability.

\medskip
\noindent
Actor-critic models combine the benefits of actor-only and critic-only models as far back as 2002 \cite{orig_a3c}. The approximation of both the actor's policy and the critic value turns out to be an effective method to control the variance which occurs in basic policy gradient algorithms like REINFORCE. The policy gradient nature approach of actor-critic also ensures that the training is much smoother when optimizing the policy space along with a value estimation compared to a pure value estimation like Q-learning. The ability to execute multiple environments and learn asynchronously means that we get a huge performance boost to regular RL frameworks. Hence we also consider implementing the asynchronous advantage actor critic model as seen in \cite{A3C}.